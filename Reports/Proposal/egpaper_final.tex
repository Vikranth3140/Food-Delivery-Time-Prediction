\documentclass[10pt,twocolumn,letterpaper]{article}

\usepackage{cvpr}
\usepackage{times}
\usepackage{epsfig}
\usepackage{graphicx}
\usepackage{amsmath}
\usepackage{amssymb}

\usepackage[breaklinks=true,bookmarks=false]{hyperref}

\cvprfinalcopy

\def\cvprPaperID{****} % *** Enter the CVPR Paper ID here
\def\httilde{\mbox{\tt\raisebox{-.5ex}{\symbol{126}}}}

\setcounter{page}{1}
\begin{document}

%%%%%%%%% TITLE
\title{Food Delivery Time Prediction: Machine Learning Project}

\author{
Vikranth Udandarao\\
Computer Science \& Engineering Dept. \\
IIIT-Delhi, India \\
{\tt\small vikranth22570@iiitd.ac.in}
\and
Mohammad Ayaan \\
Computer Science \& Engineering Dept. \\
IIIT-Delhi, India \\
{\tt\small ayaan22302@iiitd.ac.in}
\and
Swara Parekh \\
Computer Science \& Engineering Dept. \\
IIIT-Delhi, India \\
{\tt\small swara22524@iiitd.ac.in}
\and
Ananya Garg \\
Computer Science \& Engineering Dept. \\
IIIT-Delhi, India \\
{\tt\small ananya22068@iiitd.ac.in}
}

\maketitle
%\thispagestyle{empty}


%-------------------------------------------------------------------------
\section{Motivation}
Running a food delivery service comes with the challenge of keeping customers happy by delivering their meals on time and in condition despite hurdles like traffic or bad weather which can throw off the schedule unpredictably.

In order to address this issue effectively we are working on a Food Delivery Time Prediction System that utilizes machine learning methods. Our goal is to predict delivery times with precision by examining delivery data, current traffic situations and real time weather trends.

%-------------------------------------------------------------------------
\section{Related work - Research Papers}
1. \textbf{DergiPark - Comparative Analysis of ML models}: It explores machine learning models for predicting food delivery times. The study highlights temperature, traffic, and courier background as important factors and suggests future work incorporating additional features.
\textcolor{blue}{\href{https://dergipark.org.tr/en/pub/aita/issue/84471/1459560}{Link}}

2. \textbf{DergiPark - Application of Random Forest algorithm}: It analyzes online food ordering data to predict food delivery status using the Random Forest algorithm. It also suggests that traffic conditions have minimal impact on delivery timing, likely due to the use of single-person vehicles. \textcolor{blue}{\href{https://dergipark.org.tr/en/pub/forecasting/issue/60291/842180}{Link}}

3. \textbf{Medium - Food Delivery Time Prediction Model}: It emphasizes feature engineering to extract valuable insights for improved model performance. The second part focuses on further data preprocessing, including Label Encoding and Standardization, followed by training and evaluating the model.
\textcolor{blue}{\href{https://medium.com/@salonijhalani.sj/food-delivery-time-prediction-model-77200d394f2b}{Link}}

%-------------------------------------------------------------------------
\section{Timeline}
\textbf{Week 1-2}: Clean and transform data, conduct data analysis, visualize data patterns, and understand dataset characteristics.

\textbf{Week 3-4}:  Apply initial ML models and assess models using metrics like accuracy, precision, and recall.

\textbf{Week 5-6}: Refine models based on feedback and integrate new features for performance improvement.

\textbf{Week 7-8}: Test and implement different modeling techniques and feature combinations to enhance results.

\textbf{Week 8-9}: Write initial drafts for report sections, summarize model performance and findings in the report, and prepare visualizations to support conclusions.

\textbf{Week 10-11}: Complete the report with a detailed conclusion and revisions, and ensure all sections are coherent and polished. Prepare for the final project presentation

%-------------------------------------------------------------------------
\section{Individual Tasks}
- \textbf{Vikranth Udandarao}: Data collection, preprocessing, and analysis. Implementation of machine learning models.

- \textbf{Swara Parekh}: Data collection, preprocessing, and analysis. Implementation of machine learning models.

- \textbf{Mohammad Ayaan}: Evaluation of machine learning models. Optimization based on performance metrics.

- \textbf{Ananya Garg}: Data Analysis, Implementation of machine learning models, Integration of final model into a CLI.

%-------------------------------------------------------------------------
\section{Final Outcome}
The final outcome will be a Food Delivery Time Prediction system accessible using a Command Line Interface. It will predict the estimated delivery time for users based on various factors such as distance, restaurant preparation time, traffic conditions, and order details. The system will utilize machine learning algorithms like regression models to improve prediction accuracy. The project will include a detailed report and presentation showcasing the methodology, models, evaluation results, and the final integrated system.

\end{document}
